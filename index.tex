% Options for packages loaded elsewhere
\PassOptionsToPackage{unicode}{hyperref}
\PassOptionsToPackage{hyphens}{url}
\PassOptionsToPackage{dvipsnames,svgnames,x11names}{xcolor}
%
\documentclass[
  letterpaper,
  DIV=11,
  numbers=noendperiod]{scrartcl}

\usepackage{amsmath,amssymb}
\usepackage{iftex}
\ifPDFTeX
  \usepackage[T1]{fontenc}
  \usepackage[utf8]{inputenc}
  \usepackage{textcomp} % provide euro and other symbols
\else % if luatex or xetex
  \usepackage{unicode-math}
  \defaultfontfeatures{Scale=MatchLowercase}
  \defaultfontfeatures[\rmfamily]{Ligatures=TeX,Scale=1}
\fi
\usepackage{lmodern}
\ifPDFTeX\else  
    % xetex/luatex font selection
\fi
% Use upquote if available, for straight quotes in verbatim environments
\IfFileExists{upquote.sty}{\usepackage{upquote}}{}
\IfFileExists{microtype.sty}{% use microtype if available
  \usepackage[]{microtype}
  \UseMicrotypeSet[protrusion]{basicmath} % disable protrusion for tt fonts
}{}
\makeatletter
\@ifundefined{KOMAClassName}{% if non-KOMA class
  \IfFileExists{parskip.sty}{%
    \usepackage{parskip}
  }{% else
    \setlength{\parindent}{0pt}
    \setlength{\parskip}{6pt plus 2pt minus 1pt}}
}{% if KOMA class
  \KOMAoptions{parskip=half}}
\makeatother
\usepackage{xcolor}
\setlength{\emergencystretch}{3em} % prevent overfull lines
\setcounter{secnumdepth}{-\maxdimen} % remove section numbering
% Make \paragraph and \subparagraph free-standing
\makeatletter
\ifx\paragraph\undefined\else
  \let\oldparagraph\paragraph
  \renewcommand{\paragraph}{
    \@ifstar
      \xxxParagraphStar
      \xxxParagraphNoStar
  }
  \newcommand{\xxxParagraphStar}[1]{\oldparagraph*{#1}\mbox{}}
  \newcommand{\xxxParagraphNoStar}[1]{\oldparagraph{#1}\mbox{}}
\fi
\ifx\subparagraph\undefined\else
  \let\oldsubparagraph\subparagraph
  \renewcommand{\subparagraph}{
    \@ifstar
      \xxxSubParagraphStar
      \xxxSubParagraphNoStar
  }
  \newcommand{\xxxSubParagraphStar}[1]{\oldsubparagraph*{#1}\mbox{}}
  \newcommand{\xxxSubParagraphNoStar}[1]{\oldsubparagraph{#1}\mbox{}}
\fi
\makeatother


\providecommand{\tightlist}{%
  \setlength{\itemsep}{0pt}\setlength{\parskip}{0pt}}\usepackage{longtable,booktabs,array}
\usepackage{calc} % for calculating minipage widths
% Correct order of tables after \paragraph or \subparagraph
\usepackage{etoolbox}
\makeatletter
\patchcmd\longtable{\par}{\if@noskipsec\mbox{}\fi\par}{}{}
\makeatother
% Allow footnotes in longtable head/foot
\IfFileExists{footnotehyper.sty}{\usepackage{footnotehyper}}{\usepackage{footnote}}
\makesavenoteenv{longtable}
\usepackage{graphicx}
\makeatletter
\newsavebox\pandoc@box
\newcommand*\pandocbounded[1]{% scales image to fit in text height/width
  \sbox\pandoc@box{#1}%
  \Gscale@div\@tempa{\textheight}{\dimexpr\ht\pandoc@box+\dp\pandoc@box\relax}%
  \Gscale@div\@tempb{\linewidth}{\wd\pandoc@box}%
  \ifdim\@tempb\p@<\@tempa\p@\let\@tempa\@tempb\fi% select the smaller of both
  \ifdim\@tempa\p@<\p@\scalebox{\@tempa}{\usebox\pandoc@box}%
  \else\usebox{\pandoc@box}%
  \fi%
}
% Set default figure placement to htbp
\def\fps@figure{htbp}
\makeatother

\KOMAoption{captions}{tableheading}
\makeatletter
\@ifpackageloaded{caption}{}{\usepackage{caption}}
\AtBeginDocument{%
\ifdefined\contentsname
  \renewcommand*\contentsname{Table of contents}
\else
  \newcommand\contentsname{Table of contents}
\fi
\ifdefined\listfigurename
  \renewcommand*\listfigurename{List of Figures}
\else
  \newcommand\listfigurename{List of Figures}
\fi
\ifdefined\listtablename
  \renewcommand*\listtablename{List of Tables}
\else
  \newcommand\listtablename{List of Tables}
\fi
\ifdefined\figurename
  \renewcommand*\figurename{Figure}
\else
  \newcommand\figurename{Figure}
\fi
\ifdefined\tablename
  \renewcommand*\tablename{Table}
\else
  \newcommand\tablename{Table}
\fi
}
\@ifpackageloaded{float}{}{\usepackage{float}}
\floatstyle{ruled}
\@ifundefined{c@chapter}{\newfloat{codelisting}{h}{lop}}{\newfloat{codelisting}{h}{lop}[chapter]}
\floatname{codelisting}{Listing}
\newcommand*\listoflistings{\listof{codelisting}{List of Listings}}
\makeatother
\makeatletter
\makeatother
\makeatletter
\@ifpackageloaded{caption}{}{\usepackage{caption}}
\@ifpackageloaded{subcaption}{}{\usepackage{subcaption}}
\makeatother

\usepackage{bookmark}

\IfFileExists{xurl.sty}{\usepackage{xurl}}{} % add URL line breaks if available
\urlstyle{same} % disable monospaced font for URLs
\hypersetup{
  pdftitle={Estimation of out-of-sample prediction error in regression},
  colorlinks=true,
  linkcolor={blue},
  filecolor={Maroon},
  citecolor={Blue},
  urlcolor={Blue},
  pdfcreator={LaTeX via pandoc}}


\title{Estimation of out-of-sample prediction error in regression}
\author{}
\date{}

\begin{document}
\maketitle


\subsection{Assumptions about
students}\label{assumptions-about-students}

When designing this lesson plan, I've assumed that students

\begin{itemize}
\tightlist
\item
  have taken introductory courses in probability, calculus, and linear
  algebra.
\item
  have taken courses in statistical learning but have mostly focused on
  inference rather than prediction.\\
\item
  are comfortable writing code in Python and have used libraries such as
  \texttt{numpy}, \texttt{pandas}, \texttt{matplotlib}, and
  \texttt{sklearn} in previous courses (e.g., to fit a linear regression
  model) but have not used them for more advanced tasks such as
  cross-validation.
\item
  are comfortable running Python code in a jupyter lab enviroment on
  their own.
\item
  are familiar with the matrix notation used in statistical learning.
\end{itemize}

\subsection{Learning outcomes}\label{learning-outcomes}

By the end of this lesson, students should be able to:

\begin{itemize}
\tightlist
\item
  Indentify the difference between inference and prediction in
  regression.\\
\item
  Understand the training/test/validation data split
\item
  Define out-of-sample prediction error in theory and identify its
  importance.\\
\item
  Use Python to estimate out-of-sample prediction error via
  cross-validation given training data .\\
\item
  Understand the Bias-Variance Trade-Off and it relationship to
  prediction error.
\end{itemize}

\subsection{Outline}\label{outline}

\begin{itemize}
\tightlist
\item
  Lecture 1: Basics

  \begin{itemize}
  \tightlist
  \item
    Regression definition
  \item
    Inference vs Prediction
  \item
    Training/test split
  \item
    Out-of-sample prediction error in regression

    \begin{itemize}
    \tightlist
    \item
      Definition
    \item
      Why do we care about estimating out-of-sample error?
    \item
      In-class discussion: can we ever make a prediction with zero
      prediction error?

      \begin{itemize}
      \tightlist
      \item
        Goals:

        \begin{itemize}
        \tightlist
        \item
          give students a chance to digest and reflect on the topics
          covers so far,
        \item
          remind them of reducible and irreducible error
        \end{itemize}
      \end{itemize}
    \end{itemize}
  \item
    Estimating the prediction error

    \begin{itemize}
    \tightlist
    \item
      Motivation
    \item
      First approach: estimating prediction error with in-sample
      (training) error

      \begin{itemize}
      \tightlist
      \item
        Example: Linear Regression

        \begin{itemize}
        \tightlist
        \item
          In-class live-coding activity: calculate and compare the
          training and test errors using different seeds.

          \begin{itemize}
          \tightlist
          \item
            Goals:

            \begin{itemize}
            \tightlist
            \item
              give students a chance to play with training and test
              errors on simple regression model
            \item
              give thema chance to observe how different they can be.
            \end{itemize}
          \end{itemize}
        \end{itemize}
      \end{itemize}
    \end{itemize}
  \item
    The Bias-Variance Trade-Off

    \begin{itemize}
    \tightlist
    \item
      Motivating example: 10th order polynomial regression
    \item
      Definition
    \end{itemize}
  \end{itemize}
\item
  Lecture 2: Cross validation

  \begin{itemize}
  \tightlist
  \item
    Estimating the prediction error (continued)

    \begin{itemize}
    \tightlist
    \item
      Splitting the training data to get a validation set
    \item
      Second approach: A single held-out point

      \begin{itemize}
      \tightlist
      \item
        In-class discussion: Discussion of estimate for the
        out-of-sample prediction error via a single held-out point

        \begin{itemize}
        \tightlist
        \item
          Goals:

          \begin{itemize}
          \tightlist
          \item
            Give students a chance to digest the idea of a validation
            set
          \item
            Make them more comfortable with estimating the prediction
            error with training data
          \end{itemize}
        \end{itemize}
      \end{itemize}
    \item
      Third approach: Cross-validation

      \begin{itemize}
      \tightlist
      \item
        LOOCV
      \item
        K-fold cross-calidation
      \end{itemize}
    \item
      Implementing cross-valication in Python

      \begin{itemize}
      \tightlist
      \item
        How to perform LOOCV in Python
      \item
        How to perform \(K\)-fold cross-validation in Python
      \end{itemize}
    \end{itemize}
  \item
    Bias-Variance Trade-Off for k-Fold cross-validation
  \end{itemize}
\end{itemize}




\end{document}
